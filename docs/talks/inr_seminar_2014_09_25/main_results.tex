\section{Results of coupled calculations}

\subsection{Model}
\begin{frame}[fragile]
    \frametitle{3x3 minicore}
    \begin{itemize}
        \item Based on NEA PWR transient benchmark
        \item 6 UOX and 3 MOX 17x17 assemblies, 20 axial layers
        \item UOX has 2 types of rods: fuel pins and water channels
        \item MOX has 5 types of rods: 3 fuel pins, water channels and WABA
        \item Can be represented as single bundle with 51x51 rods
        \item To show performance of codes, provide benchmark for comparison
    \end{itemize}
    \begin{columns}
        \column{0.33\textwidth}
        \includegraphics[width=\textwidth]{examples/a_model_z.pdf}
        \column{0.33\textwidth}
        \includegraphics[width=\textwidth]{examples/a_model_zu.pdf}
        \column{0.33\textwidth}
        \includegraphics[width=\textwidth]{examples/a_model_zm.pdf}
    \end{columns}

\end{frame}


\subsection{Calculations}
\begin{frame}[fragile]
    \frametitle{Calculations}
    \begin{itemize}
        \item IC2 cluster, 1 node with 16 cores
        \item Standard meshtally to compute heat deposition

            \begin{bashcode}
        Iteration        kcode                 wall time    Num of cells            
                0       500000  1.0  30  100     0:05:27            3544 
                1       809016  1.0  30  100     1:38:43          107634
                2      1096763  1.0  30  100     1:52:51          107634
                3      1374895  1.0  30  100     2:06:26             ...  
                4      1647439  1.0  30  100     2:19:16          
                5      1916300  1.0  30  100     2:29:45             ... 

               20      5549120  1.0  30  100     5:05:40             ...
               21      5804749  1.0  30  100     5:15:52 
               22      6060130  1.0  30  100     5:27:00 
               23      6315284  1.0  30  100     5:38:33 
               24      6570230  1.0  30  100     5:49:10 
               25      6824985  1.0  30  100     6:00:09             ...
               26      7079562  1.0  30  100     6:12:28          107634

            \end{bashcode}
        \item MCNP initialization time 15 -- 50 min

    \end{itemize}
\end{frame}

\subsection{Results}

\begin{frame}[fragile]
    \frametitle{Heat deposition}
    \includegraphics[width=\textwidth]{examples/map_anton_e26_k01k08k19_Power.pdf}

    \includegraphics[width=\textwidth]{examples/map_anton_e26_i09i26j09j26j43_Power.pdf}
\end{frame}

\begin{frame}[fragile]
    \frametitle{Fuel temperature}
    \includegraphics[width=\textwidth]{examples/map_anton_e26_k01k08k19_Tfuel.pdf}

    \includegraphics[width=\textwidth]{examples/map_anton_e26_i09i26j09j26j43_Tfuel.pdf}
\end{frame}

\begin{frame}[fragile]
    \frametitle{Comparison with other calculations}

    \begin{itemize}
        \item Solution 1: Internal MCNP-SCF coupling (A. Ivanov, INR)
        \item Solution 2: PIRS-based 
        \item Solution 3: Internal Serpent2-SCF coupling (M. Daeubler, INR)
        \item Solution 4: Internal MCNP-SCF coupling (E. Hoogenboom, DNC)
    \end{itemize}

\end{frame}

\begin{frame}[fragile]
    \frametitle{K-eff and statistics}
    \begin{columns}
        \column{0.6\textwidth}
        \includegraphics[width=\textwidth,page=22,trim=0.8in 4.0in 1.0in 1.6in,clip=true]{examples/benchmark_.pdf}
        \column{0.4\textwidth}{\scriptsize
        \begin{itemize}
            \item Difference in K-eff about 200 pcm (except solution 4 with lower statistics)
            \item Different relaxation schemes
            \item Fuel temperature converged to less than 0.2\%
        \end{itemize}
        }
    \end{columns}
\end{frame}

\begin{frame}[fragile]
    \frametitle{Heat deposition}
    \begin{columns}
        \column{0.6\textwidth}
        \includegraphics[width=\textwidth,page=25,trim=1.2in 2.6in 0.65in 5.6in,clip=true]{examples/benchmark_.pdf}
        \column{0.4\textwidth}{\scriptsize
        \begin{itemize}
            \item Shift to lower part: coolant density (slowing down and leakage) and fuel temperature (U8 capture).
            \item Solution 4 has the strongest power, Tf and density axial shift.
            \item Tf axial shift more pronounced in S2, while density axial
                shift -- in S3. Since power shift is stronger in S2,  Tf has
                larger impact onto power axial profile in comparison to
                density.
        \end{itemize}
        }
    \end{columns}
\end{frame}

\begin{frame}[fragile]
    \frametitle{Fuel temperature}
    \begin{columns}
        \column{0.7\textwidth}
        \includegraphics[width=\textwidth,page=25,trim=1.2in 6.6in 0.65in 1.6in,clip=true]{examples/benchmark_.pdf}
        \column{0.3\textwidth}{\scriptsize
        \begin{itemize}
            \item Average values within 0.3 K (1. K for s.4)
            \item Max. values within 20 K
            \item Tfuel shifted to lower part
            \item U assemblies are about 50 K hotter. Difference grows from solution 1 to 4.
        \end{itemize}
        }
    \end{columns}
\end{frame}

\begin{frame}[fragile]
    \frametitle{Solution-to-solution max. differences}
    \begin{columns}
        \column{0.6\textwidth}
        \includegraphics[width=\textwidth,page=27,trim=1.2in 2.2in 0.65in 5.8in,clip=true]{examples/benchmark_.pdf}
        \column{0.4\textwidth}{\scriptsize
        \begin{itemize}
            \item Max. difference in Tf and coolant density found between solutions 1 and 3: 26 K and 1.66 kg/m3.
            \item Colormaps for differences (not shown) demonstrate that they are systematic.
            \item Discrepancies in results still need analysis, but 
                  overall good agreement indicate that all implementations are free from coding errors.
        \end{itemize}
        }
    \end{columns}
\end{frame}


